%%%%%%%%%%%%%%%%%%%%%%%%%%%%%%%%%%%%%%%%%%%%%%%%%%%%%%%%%%%%%%%%%%%%%%%%%%%%
% AGUtmpl.tex: this template file is for articles formatted with LaTeX2e,
% Modified July 2014
%
% This template includes commands and instructions
% given in the order necessary to produce a final output that will
% satisfy AGU requirements.
%
% PLEASE DO NOT USE YOUR OWN MACROS
% DO NOT USE \newcommand, \renewcommand, or \def.
%
% FOR FIGURES, DO NOT USE \psfrag or \subfigure.
%
%%%%%%%%%%%%%%%%%%%%%%%%%%%%%%%%%%%%%%%%%%%%%%%%%%%%%%%%%%%%%%%%%%%%%%%%%%%%
%
% All questions should be e-mailed to latex@agu.org.
%
%%%%%%%%%%%%%%%%%%%%%%%%%%%%%%%%%%%%%%%%%%%%%%%%%%%%%%%%%%%%%%%%%%%%%%%%%%%%
%
% Step 1: Set the \documentclass
%
% There are two options for article format: two column (default)
% and draft.
%
% PLEASE USE THE DRAFT OPTION TO SUBMIT YOUR PAPERS.
% The draft option produces double spaced output.
%
% Choose the journal abbreviation for the journal you are
% submitting to:

% jgrga JOURNAL OF GEOPHYSICAL RESEARCH
% gbc   GLOBAL BIOCHEMICAL CYCLES
% grl   GEOPHYSICAL RESEARCH LETTERS
% pal   PALEOCEANOGRAPHY
% ras   RADIO SCIENCE
% rog   REVIEWS OF GEOPHYSICS
% tec   TECTONICS
% wrr   WATER RESOURCES RESEARCH
% gc    GEOCHEMISTRY, GEOPHYSICS, GEOSYSTEMS
% sw    SPACE WEATHER
% ms    JAMES
% ef    EARTH'S FUTURE
% ea    EARTH AND SPACE SCIENCE
%
%
%
% (If you are submitting to a journal other than jgrga,
% substitute the initials of the journal for "jgrga" below.)

\documentclass[draft,ras]{agutex}
\usepackage{graphicx}   
\usepackage{setspace}
\usepackage{amsxtra}
\usepackage{amsmath}
\usepackage{amssymb}
\usepackage{multirow}
\usepackage{bm}
\RequirePackage{lineno}
\linenumbers
% graphics path
\graphicspath{{Figs/}}

% Author names in capital letters:
\authorrunninghead{Swoboda ET AL.}

% Shorter version of title entered in capital letters:
\titlerunninghead{ISR ERRORS}

%Corresponding author mailing address and e-mail address:
%\authoraddr{Corresponding author: A. B. Smith,
%Department of Hydrology and Water Resources, University of
%Arizona, Harshbarger Building 11, Tucson, AZ 85721, USA.
%(a.b.smith@hwr.arizona.edu)}

\begin{document}

%% ------------------------------------------------------------------------ %%
%
%  TITLE
%
%% ------------------------------------------------------------------------ %%


%\title{Trade-Offs Between Statistical Accuracy and Space-Time Resolution in ISR}
\title{Observability of Ionospheric Space-Time Structure with ISR:   A simulation study }
%%%%%%%%%%%%%% Author Info %%%%%%%%%%%%%%%%%%%%%%%%%%%%%%%%%%%%%
\authors{John Swoboda,\altaffilmark{1}
Joshua Semeter,\altaffilmark{1} Philip Erickson \altaffilmark{2}}

\altaffiltext{1}{Department of Electrical \& Computer Engineering,
Boston University, Boston, Massachusetts, USA.}
\altaffiltext{2}{Haystack Observatory, Massachusetts Institute of Technology, Westford, Massachusetts, USA.}
%%%%%%%%%%%%%% Abstract %%%%%%%%%%%%%%%%%%%%%%%%%%%%%%%%%%%%%
%% ------------------------------------------------------------------------ %%
%
%  ABSTRACT
%
%% ------------------------------------------------------------------------ %%

% >> Do NOT include any \begin...\end commands within
% >> the body of the abstract.

\begin{abstract}
As with any sensing modality incoherent scatter radar (ISR) has inherent errors and uncertainty in its measurements. A number of theoretical aspects behind these errors have been documented in the literature, which leads to a trade off between spatial and temporal resolution and statistical accuracy. 

Resource allocation and experiment design. Understand degrees of freedom in design.

The recent application of phased array antennas with pulse to pulse steering allow for greater flexibility in processing along with making it is now possible to create full volumetric reconstructions of plasma parameters. These phased array systems are used heavily in the high latitude region of the ionosphere, which can have plasma phenomena that is highly variable in space and time. In order to develop an experiment to observe the plasma phenomena researchers have to wade through a large number of degrees of freedom. In order to properly allocate resources in some cases researchers may need to simulate the experiment.

This publication will show a simulator that can take a field of plasma parameters and create ISR data at the IQ level and then process it to show a possible reconstruction of the parameters field. This simulator can be used to create ISR data to test new algorithms to better reconstruct the plasma parameter field. It can also give researchers a new tool that can assist them in the set up their experiments. This simulation will overall give a full forward model description of the ISR reconstruction.
\end{abstract}

%% ------------------------------------------------------------------------ %%
%
%  BEGIN ARTICLE
%
%% ------------------------------------------------------------------------ %%

% The body of the article must start with a \begin{article} command
%
% \end{article} must follow the references section, before the figures
%  and tables.

\begin{article}

\section{Introduction}
Incoherent scatter radar is an important diagnostic for the ionosphere in that it can give direct measurements of the intrinsic plasma parameters  \citep{dougherty:farley1960, farleydougherty:ISR2, doughteryfarley:ISR3, hagfors1961}. As with all diagnostic tools it has associated with it sources of errors which include time and spatial ambiguities \citep{farley1969, farleycomppower1969, hysell2008, RDS:RDS20236}.  

One unique aspect of ISR is that inherent random fluctuations of the plasma are used to create these measurements. These fluctuations are used by creating second order statistics from a scattered signal, specifically an autocorrelation function (ACF) \citep{farley1969}. The statistical nature of the target itself yields the requirement of averaging numerous realizations of the ACF to reduce the variance of the estimate. This forces the assumption of stationarity for a space-time cell, which may not be true. In the end this creates a trade-off between space-time resolution and the variance of the measurements.

Application of electronically steerable array (ESA) technology to ISR has been a recent advancement for the community. ISRs such as these, like the Advanced Modular Incoherent Scatter (AMISR) systems, have already been deployed in Poker Flat Alaska and Resolute Bay Canada \citep{Nicolls:2007ie, dahlgren2012di}. These ESA based systems are seen as the future of the ISR sensor modality due to the flexibility in beam steering, processing and other aspects over dish based systems. The next step in the evolution of these systems is expected to be the EISCAT-3D project, which will have a number of enhancements such as multi-static processing capability and be able to receive and process data from each phased array element by default.

One benefit of ESA based ISR is that volumetric reconstructions of plasma parameters can be created \citep{Semeter2009738, Nicolls:2007ie, dahlgren2012di}. These systems also have ben used to reconstuct full vector parameters using estimates of the ion velocity which can be determined using the Doppler shift of spectra \citep{butler:imagingfregiondrifts,RDS:RDS20195}. Still it has been shown that the volumetric reconstructions can yield measurements with a high degree of ambiguity \citep{Dahlgren:2012dq}. Similar type of ambiguities have been seen when using systems with a dish antenna as well. In \citet{Semeter:2005fo} the authors to show an undersampling in the horizontal dimension, but are able to compensate by changing processing parameters.

With these new capabilities for the ISR community a discussion of the possible sources of uncertainty and error is needed. These sources of error and ambiguity though are difficult understand in the context of experiment design. With that in mind it may be useful to simulate the ISR measurement process before an experiment is attempted. With that in mind this paper will show how one could simulate an experiment, the outline of this is a follows. After listing the possible sources of error and ambiguity in ISR our simulation method will be detailed. After which  a number of examples of the simulator will be shown. These example range from a stationary column of enhanced electron density to the output of a self-consistent multi-fluid ionosphereic model \citep{semeter:plasmatransport2012}. These examples will illustrate how one could develop their experiments in a systematic way in order make measurements that best reflect the physics present in the ionosphere.

\section{ISR Errors}

In this section the main sources of ISR errors will be discussed. The first part of this discussion will cover the statistical errors that arise from the ISR process. After that the errors from the spatial and temporal ambiguity of ISR systems will be shown. This in the end will lead to trade offs that the experiment designer will have to face.

\subsection{Statistical Errors}

To measure the plasma parameters ISR takes advantage of the random fluctuations of electron density in the ionosphere. The theory of how the plasma parameters impact the statistics of these fluctuations have been discussed since the first use of this sensor modality \citep{gordon58,dougherty:farley1960, farleydougherty:ISR2, doughteryfarley:ISR3, hagfors1961}, and even as recent as 2011 there have been new formulations of this theory \citep{kudeki:milla:1,kudeki:milla:2}. 

The two main sources of statistical error will covered here are the random fluctuations from the electron density and noise from within the sensor itself. There are other sources of statistical error including sky noise and coherent scatter from other targets. 

The raw incoherent scatter signal is itself is a random process. As such it is necessary to average samples of an estimator for autocorrelation or spectrum \citep{Diaz:2008co}. An easy rule of thumb to understand how the error will reduce can be seen in \citet{farley1969},

\begin{equation}
\label{eqn:basicerror}
\left\langle \left| \hat{R}(\tau) -R(\tau) \right|^2 \right\rangle \propto \frac{1}{\sqrt{J}},
\end{equation}
\noindent where $R(\tau)$ is the ACF as a function of lag $\tau$, $\hat{R}(\tau)$ is its estimate and $J$ is the number of samples or pulses averaged together to create the estimate.

The variance of this signal is further degraded once noise is added. 

As one adds more and more pulses we can assume that the signal is Gaussian like due to the central limit theorem. 

\subsection{Space-Time Errors}

The errors created through the ambiguity function lead to a blurring or averaging of ACFs from different points in time and space. This is similar to a blurring operator one might see in a camera or numerous other types of sensors. With ISR this can be more problematic due to the non-linear fitting step.

The space-time ambiguity, $L(\tau_s,\mathbf{r}_s,t_s,\tau,\mathbf{r},t)$, is the kernel of Fredholm integral equation of the first kind operating on the ACF, $R(\tau,\mathbf{r},t)$, which can change over space, $\mathbf{r}$, and time $t$. which can be represented as follows,

 \begin{equation}
  \label{eqn:staf}
  \rho(\tau_s,\mathbf{r}_s,t_s) =\int L(\tau_s,\mathbf{r}_s,t_s,\tau,\mathbf{r},t)R(\tau,\mathbf{r},t)dVdtd\tau,
\end{equation}

\noindent where the subscript $s$ represents the same variable but now discretely sampled by the radar. 

The kernel is a separable function when the spatial coordinates are spherical, where ($r,\theta,\phi)$ represent, range, azimuth and elevation respectively. This the changes Equation \ref{eqn:staf} as follows,

\begin{equation}
\label{eqn:stafbrok}
\rho(\tau_s,\mathbf{r}_s,t_s)= \int G(t_s,t)F(\theta_s,\phi_s,\theta,\phi)W(\tau_s,r_s,\tau,r) R(\tau,\mathbf{r},t) dVdt d\tau,
\end{equation}

\noindent where $G(t_s,t)$ is the kernel for the time dimension, $F(\theta_s,\phi_s,\theta,\phi)$ is radar beam shape which acts as a kernel in azimuth and elevation, and $W(\tau_s,r_s,\tau,r) $ which is the range ambiguity function which acts as a kernel along range $r$ and lag $\tau$. The derivation of this operator can be seen in \citet{RDS:RDS20236}.

These two sources of error create a significant trade off between statistical variation of the signal and spatial and temporal resolution of the signal. In order to reduce the statical fluctuations in the signal pulses need to be averaged together. This is necessary even for the case where there is no noise, in a sense the infinite signal to noise ratio (SNR) case. This integration is mainly done over time but can be done over space as well. For phased array systems this mixture of spatial and temporal averaging can be done by averaging together beams. This though will reduce cross range resolution but could possibly improve temporal resolution. It is for this reason these types of trade offs can best be explored through simulation, which will be covered in the following sections.
 
\section{Simulator}

The following section will detail the processing steps in the ISR simulator. The first part will detail the creation of the inphase and quadrature data (IQ data). The next will detail the processing used to create the estimates of the ACFs, which will also be refered to as lag products.

\subsection{Inputs}
The simulator takes as input a discretized set of ionosphere parameters in Cartesian coordinates and which can change with time. Each point in time and space has a set of parameters that allow it to make an ISR spectrum using the methods detailed in the previous chapter. The spectrums are then created so every point in space and time will have its own intrinsic ISR spectrum. The radar will then act on these spectrums as a linear operator and average them together in time and space using the beam patterns and pulse pattern. 

Using the fact that any spatial correlations between the electron density fluctuations will be on the order of the Debye length \cite{farley1969}, the intrinsic ISR spectrums will be first averaged over a resolution cell for the radar. 

\subsection{ IQ Data Creation}
The IQ data is created by taking a complex white Gaussian noise process and shaping the spectrum using a filter. Each point in space and time will have a separate noise plant and filter which is derived from the plasma and radar parameters parameters, like that seen in Figure \ref{fig:IQdiagram}. 
%The 3-D ISR simulator creates data by deriving a time filter from the autocorrelation functions and applying them to complex white Gaussian noise generators. Stating this in another way, every point in time and space has a noise plant and filter structure as in Figure \ref{fig:IQdiagram}. 
\begin{figure}[h!]
\centering
\includegraphics[width=4in]{diagrampart}
\caption{Diagram for I/Q simulator signal flow.}
\label{fig:IQdiagram}
\end{figure}

The radar samples the space in a spherical coordinate system with discrete range and beam positions. For each range gate and beam the different spectrums are averaged together together. In range this is simply a window the length of a range gate. Across the azimuth and elevation space the beam pattern for the system is used. In order to calculate the beam pattern for the AMISR system the method detailed in the appendix of \citep{RDS:RDS20236}. The entire process of the spatial sample is shown in the simplified diagram in Figure \ref{fig:beamdia}.


\begin{figure}[!t]
\centering
\includegraphics[width=2in]{beamsampling}
% where an .eps filename suffix will be assumed under latex, 
% and a .pdf suffix will be assumed for pdflatex; or what has been declared
% via \DeclareGraphicsExtensions.
\caption{Beam Sampling Diagram}
\label{fig:beamdia}
\end{figure}

Once the spectrum has been created the filter, $H_m(\omega)$, is created by simply taking the square root of the spectrum, $S_m(\omega | \: \bm{\theta})$

\begin{equation}
\label{eq1}
H_m(\omega) = \sqrt{S_m(\omega | \: \bm{\theta})}.
\end{equation}

\noindent The term $ \bm{\theta}$ refers to the different plasma and system parameters needed to make the spectrum. Complex white Gaussian noise, CWGN, ($w(k)\sim CN(0,\mathbf{I})$) is then pushed though each of the filters and then windowed by the pulse creating the following:   

\begin{equation}
\label{eq2}
y_m (k)= s(k)\left[h_m(k)*w(k)\right],
\end{equation}
 
\noindent where $s(k)$ is the pulse shape. The application of this filter is actually done in the frequency domain. This is possible because the Discrete Fourier Transform (DFT) of a vector of CWGN is also CWGN. The only difference is that there is a change in the variance, which is tied to the number of points used in the DFT \citep{kayvol1}. With this in mind Equation \ref{eq2} can be implemented as the following,

\begin{equation}
\label{eq:fftfilt}
y_m (k)= s(k)\displaystyle \sum_{i=0}^{K-1}e^{j\omega_ik}\left[ \sqrt{S_m(\omega_i | \: \bm{\theta})}w(\omega_i)\right],
\end{equation}

\noindent where $\omega_i$ is the frequency variable, $w(\omega_i) \sim CN(0,\mathbf{I})$ and $K$ is the number of points used for the DFT \citep{michellnoisesim1981}.

\begin{figure}[!h]
\centering
\includegraphics[width=7.0in]{diagram}
\caption{ISR Simulation Diagram}
\label{fig:isrdiag}
\end{figure}


After the data for each range gate $y_m(k)$ is created the power of the return is calculated

\begin{equation}
\label{eq3}
P_r = \frac{cG \lambda^2}{2(4\pi)^2}\frac{P_t }{R^2}\frac{\sigma_e N_e}{(1+k^2\lambda_D^2),(1+k^2\lambda_D^2 + T_r)}
\end{equation}
 
 \noindent where $P_r$ is the power received, $c$ is the speed of light, $G$ is the gain of the antenna, $P_t$ is the power of the transmitter, $\sigma_e$ is the electron radar cross section, $k$ is the wavenumber of the radar, $\lambda_D$ is the Debye length, $N_e$ is the electron density and $T_r$ is the electron to ion temperature ratio.
  
Once the power has been calculated for each range all of the data is delayed and summed together so as to model the arrival of the radar return at the receive: 
 
\begin{equation}
\label{eq4}
x(n) = \displaystyle\sum\limits_{m =0}^{M-1} \alpha(m)y_m(n-m),
\end{equation}

\noindent where $\alpha(m) = \sqrt{P_r(m)}/\hat{\sigma_y}$ and $\hat{\sigma_y}$ is the estimate of the standard deviation of $y_m(k)$. Lastly, to model the inherent noise in the radar and environment more complex Gaussian noise is added

\begin{equation}
\label{eq4}
x_f(n) = x(n) +\sqrt{\frac{k_bT_{sys}B}{2}} w(n), \quad w(n)\sim CN(0,1)
\end{equation}

\noindent where $k_b$ is Boltzmann's constant, $T_{sys}$ is the system temperature and $B$ is the system bandwidth.
A full diagram of the model can be seen in Figure \ref{fig:isrdiag}.

%%%%%%%%%%%%%%%%%%%%%%%%%%%%%%%%%%%%%%%%Chapter 3%%%%%%%%%%%%%%%%%%%%%%%%%%%%%%%%%%%%%%%%%%%%%%%%%%%
\subsection{ACF Estimation}

After the IQ data has been created it is processed to create estimates of the ACF at desired points of space. This type of processing has been detailed and analyzed in \citep{farley1969} and in other publications. This processing follows a flow chart seen in Figure \ref{fig:chain}.

\begin{figure}[!t]
\centering
\includegraphics[width=6in]{datastackchain}
\caption{ISR signal processing chain, with signal processing operations as squares and data products as diamonds.}
\label{fig:chain}
\end{figure}


The lag product formation is an initial estimate of the autocorrelation function. The sampled I/Q can be represented as $x(n) \in\mathbb{C}^N$ where $N$ is the number of samples in an inter pulse period. For each range gate $m\in 0,1,...M-1$ an autocorrelation is estimated for each lag of $l \in 0,1...,L-1$.  To get better statistics this operation is performed for each pulse $j\in 0,1,...J-1$ and then summed over the $J$ pulses. The entire operation to form the initial estimate of $\hat{R}(m,l)$ can be seen in Equation \ref{lagpro}:

\begin{equation}
\label{lagpro}
\hat{R}(m,l) = \displaystyle\sum\limits_{j=0}^{J-1} x(m-\lfloor l/2\rfloor,j)x^*(m+\lceil l/2 \rceil,j).
\end{equation}

The case shown in Equation \ref{lagpro} is a centered lag product, other types of lag products calculations are available but generally a centered product is used. In the centered lag product case range gate index $m$ and sample index $n$ can be related by $m=n-\lfloor L/2\rfloor$ and the maximum lag and sample relation is $M=N-\lceil L/2 \rceil$.  This lag product formation is the first step in taking a discrete Wigner Distribution \citep{TFAcohen}.

This specific type of lag product formation is detailed in \citep{farley1969} and had been referred to as unbiased. This terminology does differ from what is used in statistic signal processing literature such as \citep{randomsigshanmugan} where the unbiased autocorrelation function estimate is carried out as so,

\begin{equation}
\label{eq:lagproub}
\hat{R}(m,l) = \frac{1}{L-l}\displaystyle\sum\limits_{j=0}^{J-1} x(m-\lfloor l/2\rfloor,j)x^*(m+\lceil l/2 \rceil,j).
\end{equation}

\noindent With out the $\frac{1}{L-l}$ term the estimator will be windowed with a triangular function thus impacting the estimate of the ISR spectrum as this will act as a convolution in the frequency domain. This bias is taken into account in \citep{farley1969} but it is simply wrapped up into the ambiguity function. 

Applying a summation rule is usually the next step in creating an estimate of the autocorrelation function.  This is done to get a constant range ambiguity across all of the lags for long pulse experiment\citep{nygren1996}. It also equalizes the statistics for each lag as the higher lags have greater variance. 

An example summation rule for a forward product is shown in Figure \ref{fig:sumrule}. In the figure the image on the left is a basic representation of an ambiguity function of a long pulse.  Its mirrored on the right with red bars which would show the integration area under it so the ambiguity function will be of equal size in range.  

In the processing this is basically a summing of lags from different ranges.  The amount of summing is similar to what is shown in Figure \ref{fig:sumrule}.  There are a number of different summing rule each with their own trade offs \citep{nygren1996}.  

Lastly an estimate of the noise correlation is subtracted out of $\hat{R}(m,l)$, which is defined as $\hat{R}_w(m,l)$:

\begin{equation}
\label{lagpro}
\hat{R}_w(m,l) = \displaystyle\sum\limits_{j=0}^{J-1} w(m_w-\lfloor l/2\rfloor,j)w^*(m_w+\lceil l/2 \rceil,j),
\end{equation}

\noindent where $w(n_w)$ is the background noise process of the radar.  Often the noise process is sampled during a calibration period for the radar when nothing is being emitted.  The final estimate of the autocorrelation function after the noise subtraction and summation rule will be represented by $\hat{R}_f(m,l)$.
\begin{figure}[!t]
\centering
\includegraphics[width=3in]{sumrule}
% where an .eps filename suffix will be assumed under latex, 
% and a .pdf suffix will be assumed for pdflatex; or what has been declared
% via \DeclareGraphicsExtensions.
\caption{Summation Rule Diagram}
\label{fig:sumrule}
\end{figure}

After the final estimation of the spectrum is complete the nonlinear least squares fitting takes place to determine the parameters.  The basic class of nonlinear least-squares problems as seen in \citep{kayvol1}, are shown in Equation \ref{nlls},

\begin{equation}
	\hat{\mathbf{p}}= \underset{\mathbf{p}}{\text{argmin}} (\mathbf{y}-\bm{\theta}(\mathbf{p}))^*\bm{\Sigma}^{-1}(\mathbf{y}-\bm{\theta}(\mathbf{p})).
\label{nlls}
\end{equation}

In Equation \ref{nlls}, the data represented as $\mathbf{y}$ would be the final estimate of the autocorrelation function $\hat{R}_f(m,l)$ at a specific range or its spectrum $\hat{S}_f(m,\omega)$.  The parameter vector $\mathbf{P}$ would be the plasma parameters $N_e$, $T_e$, $T_i$ and various other parameters including ion velocities. The fit function $\bm{\theta}$ is the IS spectrum calculated from models, such as once seen in \citep{kudeki:milla:1}, smeared by the ambiguity function.  In the case of the long pulse the ambiguity can be simply applied by multiplying it with the autocorrelation function $R(l)$, if the summation rule is properly applied. The correlation matrix $\bm{\Sigma}$ is often realized as a diagonal matrix for many ISR systems the variance of the lags or each point of the spectrum being the values. The variance of the ACF estimator can be estimated using the following,

\begin{equation}
\label{eqn:acfvar}
\sigma_{\hat{R}(l)}^2=\frac{1}{JL}\displaystyle \sum_{m=-(L-l-1)}^{L-l-1}\left(\frac{L-|m|+1}{L}\right)\left(|\hat{R}(m)|^2 +|\hat{R}(m+l)\hat{R}(m-l)|\right) + \hat{N}^2
\end{equation}

\noindent where $N$ is the estimated noise power. To estimate the spectrum variance the matrix $\bm{\Sigma}$ is transformed in to the Fourier domain using FFTs (FFT on the columns and IFFT on the rows) so as to model the $\mathbf{F}\bm{\Sigma} \mathbf{F}^*$ matrix operation. 



%%
%The diagonal values usually used, noted as $\sigma_i^2$, usually the same unless there is a larger measurement error for one of the lags or spectrums.  The following formula from  \citep{nicollsisrschool2013} can be used:

%\begin{equation}
%\label{sigpow}
%\sigma_i = \frac{S}{\sqrt{J}}\left(1+\frac{1}{SNR}\right).
%\end{equation}

%\noindent where $S$ is the signal power and $SNR$ is the signal to noise ratio. The noise level can be estimated from the calibration period. 

In the past ISR researchers have used the Levenberg-Marquart algorithm to fit data \citep{nikoukar2008}.  This specific iterative algorithm moves the parameter vector $\mathbf{p}$ by a perturbation $\mathbf{h}$ at each iteration\citep{gavin:2013}.  Specifically Levenberg-Marquart was designed to be a sort of meld between two different methods Gradient Decent, and Gauss-Newton.  The perturbation vector $\mathbf{h}_{lm}$ can be calculated using the following:

\begin{equation}
\left[ \mathbf{J}^T\bm{\Sigma}^{-1}\mathbf{J}\right]\mathbf{h}_{lm} =\mathbf{J}^T\bm{\Sigma}^{-1}(\mathbf{y}-\bm{\theta}(\mathbf{p}))
\label{hlm}
\end{equation}

\noindent where $\mathbf{J}$ is the Jacobian matrix $\partial \bm{\theta}/\partial \mathbf{p}$ \citep{levenberg1944,marquardt:1963}. 

Using the scipy optimize tool box the fitted parameters can determined using the leastsquares function. This function outputs the fitted parameters along with a covariance matrix. This matrix is calculated using a numerical approximation to the Jacobian matrix that the function uses to determine the solution. The Hessian, $\mathbf{H}$ is then calculated by using the Jacobian and then inverted to get the covariance matrix. Due to the way the numerical routines solve the problem this matrix must be multiplied by the error between the estimated parameters and the data,

\begin{equation}
\label{eqn:jacinv}
\bm{\Sigma}_{\hat{\mathbf{p}}}=\frac{(\mathbf{J}^T\mathbf{J})^{-1} (\mathbf{y}-\bm{\theta}(\hat{\mathbf{p}}))^*\bm{\Sigma}^{-1}(\mathbf{y}-\bm{\theta}(\hat{\mathbf{p}}))}{L-N_{\mathbf{p}}},
\end{equation}

\noindent where $N_{\mathbf{p}}$ is the number of parameters being fit. The variances of the parameters are then taken as the diagonals of the matrix. Often though the Hessian matrix is undefined so it can not be inverted so the error term is then set as a NaN.


\section{Simulation Examples}

\section{Conclusion}

\begin{acknowledgments}
This work was supported by the National Science Foundation, through Aeronomy Program Grant AGS-1339500 to Boston University and Cooperative Agreement AGS-1242204 between the NSF and the Massachusetts Institute of Technology, and by the Air Force Office of Scientific Research under contract FA9550-12-1-018.   The authors are grateful to the International Space Science Institute (ISSI, Bern, Switzerland) for sponsoring a series of workshops from which the idea for this work emerged. 

Software used to create figures for this publications can be found at https://github.com/jswoboda/. Please contact the corresponding author, John Swoboda at swoboj@bu.edu, with any questions regarding the software along with any requests for the specific data used for the figures. \end{acknowledgments}


\bibliographystyle{BibTeX/agufull08}
\bibliography{BibTeX/litreview}
\end{article}

\end{document}

