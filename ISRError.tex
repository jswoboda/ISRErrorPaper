%%%%%%%%%%%%%%%%%%%%%%%%%%%%%%%%%%%%%%%%%%%%%%%%%%%%%%%%%%%%%%%%%%%%%%%%%%%%
% AGUtmpl.tex: this template file is for articles formatted with LaTeX2e,
% Modified July 2014
%
% This template includes commands and instructions
% given in the order necessary to produce a final output that will
% satisfy AGU requirements.
%
% PLEASE DO NOT USE YOUR OWN MACROS
% DO NOT USE \newcommand, \renewcommand, or \def.
%
% FOR FIGURES, DO NOT USE \psfrag or \subfigure.
%
%%%%%%%%%%%%%%%%%%%%%%%%%%%%%%%%%%%%%%%%%%%%%%%%%%%%%%%%%%%%%%%%%%%%%%%%%%%%
%
% All questions should be e-mailed to latex@agu.org.
%
%%%%%%%%%%%%%%%%%%%%%%%%%%%%%%%%%%%%%%%%%%%%%%%%%%%%%%%%%%%%%%%%%%%%%%%%%%%%
%
% Step 1: Set the \documentclass
%
% There are two options for article format: two column (default)
% and draft.
%
% PLEASE USE THE DRAFT OPTION TO SUBMIT YOUR PAPERS.
% The draft option produces double spaced output.
%
% Choose the journal abbreviation for the journal you are
% submitting to:

% jgrga JOURNAL OF GEOPHYSICAL RESEARCH
% gbc   GLOBAL BIOCHEMICAL CYCLES
% grl   GEOPHYSICAL RESEARCH LETTERS
% pal   PALEOCEANOGRAPHY
% ras   RADIO SCIENCE
% rog   REVIEWS OF GEOPHYSICS
% tec   TECTONICS
% wrr   WATER RESOURCES RESEARCH
% gc    GEOCHEMISTRY, GEOPHYSICS, GEOSYSTEMS
% sw    SPACE WEATHER
% ms    JAMES
% ef    EARTH'S FUTURE
% ea    EARTH AND SPACE SCIENCE
%
%
%
% (If you are submitting to a journal other than jgrga,
% substitute the initials of the journal for "jgrga" below.)

\documentclass[draft,ras]{agutex}
\usepackage{graphicx}   
\usepackage{setspace}
\usepackage{amsxtra}
\usepackage{amsmath}
\usepackage{amssymb}
\usepackage{multirow}
\usepackage{bm}
\RequirePackage{lineno}
\linenumbers
% graphics path
\graphicspath{{Figs/}}

% Author names in capital letters:
\authorrunninghead{Swoboda ET AL.}

% Shorter version of title entered in capital letters:
\titlerunninghead{ESA ISR ERRORS}




\begin{document}

%% ------------------------------------------------------------------------ %%
%
%  TITLE
%
%% ------------------------------------------------------------------------ %%


%\title{Trade-Offs Between Statistical Accuracy and Space-Time Resolution in ISR}
\title{Observability of Ionospheric Space-Time Structure with ISR:   A Simulation Study }
%%%%%%%%%%%%%% Author Info %%%%%%%%%%%%%%%%%%%%%%%%%%%%%%%%%%%%%
\authors{John Swoboda,\altaffilmark{1}
Joshua Semeter,\altaffilmark{1} Matthew Zettergren,  \altaffilmark{2} Philip J. Erickson, \altaffilmark{3}}

\altaffiltext{1}{Department of Electrical \& Computer Engineering,
Boston University, Boston, Massachusetts, USA.}
\altaffiltext{2}{Physical Sciences Department, Embry-Riddle Aeronautical University, Daytona Beach, Florida, USA.}
\altaffiltext{3}{Haystack Observatory, Massachusetts Institute of Technology, Westford, Massachusetts, USA.}
%%%%%%%%%%%%%% Abstract %%%%%%%%%%%%%%%%%%%%%%%%%%%%%%%%%%%%%
%% ------------------------------------------------------------------------ %%
%
%  ABSTRACT
%
%% ------------------------------------------------------------------------ %%

% >> Do NOT include any \begin...\end commands within
% >> the body of the abstract.

\begin{abstract}

%MZ - "this data" --> "these data" in the abstract and elsewhere?

The sources of error from electronically steerable array (ESA) incoherent scatter radar (ISR) systems are investigated both theoretically and with use of an open source ISR simulator, developed by the authors, called Simulator for ISR (SimISR). The main sources of error incorporated in the simulator include statistical uncertainty, which arises due to nature of the measurement mechanism, and the inherent space-time ambiguity from the sensor. SimISR can take a field of plasma parameters, parameterized by time and space, and create simulated ISR data at the scattered electric field (i.e. complex receiver voltage) level, subsequently processing this data to show possible reconstructions of the original parameter field. To demonstrate general utility, we show a number of simulation examples, with two cases using data from a self-consistent multi-fluid transport model. Results highlight the significant influence of the forward model of the ISR process and the resulting statistical uncertainty on plasma parameter measurements, and the core experiment design trade offs that must be made when planning observations. These conclusions further underscore the utility of this class of measurement simulator as a design tool for more optimal experiment design efforts using flexible ESA class ISR systems.

%MZ - We're in the process of revising our first paper with a new Bi-Maxwellian model.  Just curious - could you include temperature anisotropy effects on the spectra and fitting process?

\end{abstract}

%% ------------------------------------------------------------------------ %%
%
%  BEGIN ARTICLE
%
%% ------------------------------------------------------------------------ %%

% The body of the article must start with a \begin{article} command
%
% \end{article} must follow the references section, before the figures
%  and tables.

\begin{article}

\section{Introduction}

% MZ - Not sure about the use of the word "direct" in the first sentence

Incoherent scatter radar (ISR) is a critical technology for ionospheric research since it yields direct measurements of intrinsic plasma parameters  \citep{dougherty:farley1960, farleydougherty:ISR2, doughteryfarley:ISR3, hagfors1961}. As with all diagnostic tools it has unavoidable measurement uncertainties, including temporal and spatial ambiguities \citep{farley1969,farleycomppower1969,Woodman:1991is,hysell2008,RDS:RDS20236}.  ISR uses the inherent random fluctuations of the plasma as a remote sensing diagnostic. These fluctuations are processed for information extraction by computing second order statistics, specifically an autocorrelation function (ACF), from the scattered signal \citep{farley1969}. The stochastic nature of the target itself yields the requirement of averaging multiple realizations of the ACF to compute a statistical ensemble average, with more averages reducing the variance of the estimate. This forces the assumption of wide-sense stationarity for a space-time cell, which may not always be valid. In the context of experiment design, this creates a trade-off between space-time resolution and the variance of the measurements.

Application of electronically steerable array (ESA) technology to ISR has been a recent community advancement \citep{kelly:pfisr2009}. ESA capable ISRs, such as the Advanced Modular Incoherent Scatter (AMISR) systems, have already been deployed in Poker Flat Alaska and Resolute Bay Canada \citep{Nicolls:2007ie, dahlgren2012di}. ESA based systems are seen as the future of the ISR sensor modality due to the flexibility in beam steering and processing over single antenna based systems. The next step in the evolution of these systems is the EISCAT-3D project, which is expected to have a number of enhancements such as multi-static processing capability \citep{turunen:eiscat3d2009}.

One benefit of ESA-based ISR is that volumetric reconstructions of plasma parameters can be created \citep{Semeter2009738, Nicolls:2007ie, dahlgren2012di}. ESA systems also have been used to reconstruct full vector velocity fields from line-of-sight bulk Doppler measurements \citep{butler:imagingfregiondrifts,RDS:RDS20195}; however, it has been shown that reconstructions of this type can yield measurements with a high degree of parameter ambiguity \citep{Dahlgren:2012dq}. Similar types of ambiguities also arise using systems employing a single high gain antenna.  \citet{Semeter:2005fo} illustrated the challenges in interpreting spatially scanned measurements in a highly dynamic environment, as well as the benefits of a flexible post-processing environment.

With the continued interest in ISR as an imaging sensor, a rigorous mathematical framework is needed to understand uncertainties in the high-level data products produced.  These sources of error, from statistical uncertainty and spatio-temporal ambiguity, create complex functions and choices in the context of experiment design. Because of this complex multi-dimensional trade space, it is useful to simulate the ISR measurement process before an experiment is attempted.  ISR is also increasingly called upon to validate predictions from multi-dimensional physical models.  Simulation has a critical role in this context by allowing us to assess the observability of a predicted  response, which may require detection of subtle variations in ionospheric state parameters.   In general terms, simulation can provide a more comprehensive error analysis beyond what is currently contained in a simple estimate of error bars. 

This paper describes a method to simulate the entire ISR process with consideration of all three spatial sampling dimensions for a monostatic radar. This method has been implemented in a software package called SimISR (Simulator for ISR).  
% JOHN:  The acronym is unpronounceable, and the sequence of letters hard to say.  Are you sure you want to stick with this?  Could do something simpler, like ISRSim, or SimISR (pronounced "simizer", sort of like "amizer")
% MZ - Call it a "new" software package?
The remainder of paper is organized as follows. Section 2 will list possible sources of ISR error as included in our simulation method. The simulation methodology behind SimISR will be described in Section 3. Section 4 will show a number of plasma parameter use cases conducted with the simulator, ranging from a stationary column of enhanced electron density to the output of a self-consistent multi-fluid ionospheric model \citep{semeter:plasmatransport2012}. 
% MZ - The 3D results actually come from a totally different codebase.  We've only published one other paper with this code:
%Zettergren, M. D., J. L. Semeter, and Hanna Dahlgren. "Dynamics of density cavities generated by frictional heating: %Formation, distortion, and instability." Geophysical Research Letters 42, no. 23 (2015).
%Harvard	
These examples point the way towards development of best practices for systematic ESA class ISR experiment designs that are intended to conduct measurements best reflecting the plasma parameter state actually present in the ionosphere.

%%%%%%%%%%%%%%%%%%%%%%%%%%%%%%%%%%%%%%%%%%%%%%%%%%%%%%%%%%%%%%%%%%%%%%%%%%%%%%%%%%%%%%%%%%%%%%
\section{ISR Errors}

In this section two sources of ISR error are discussed. The first part of this discussion covers the statistical uncertainty that arises from the ISR measurement process. Following this, the errors originating from both the spatial and temporal ambiguity of ISR experimental systems are described. The goal is to develop a more fully realized foundation for the trade-offs that must be considered by the ISR experiment designer for a given science objective.

\subsection{Statistical Errors}

The electron density fluctuations observed by ISR techniques are associated primarily with two longitudinal wave modes in the plasma: the ion-acoustic mode and the Langmuir mode \citep{evans;isr}.  Most ISR experiments involve the use of multiple independent samples, created by measuring ionospheric scatter from repeated individual radar pulses, to create an estimate of the ion acoustic spectrum (i.e., the Fourier transform of the autocorrelation function, or ACF). This experimental observation is then fit to a theoretical plasma physics based model in order to estimate ionospheric state parameters (typically plasma density, electron temperature, ion temperature, and bulk line-of-sight plasma motion). The theory of how the plasma parameters impact the statistics of density fluctuations has been developed in a series of studies \citep{gordon58,dougherty:farley1960, farleydougherty:ISR2, doughteryfarley:ISR3, hagfors1961}, with reformulations of the theory occurring even recently  \citep{kudeki:milla:1,kudeki:milla:2}. 

The two main sources of statistical uncertainty in the received voltages measured at the antenna terminals are described here as (1) random fluctuations from the electron density, and (2) noise from astronomical sources and from  the sensor itself. There are other sources of increased uncertainty, including coherent scatter from other targets, but these will be assumed negligible in this study.  The received voltages comprising the incoherent scatter signal also have the properties of a random process; as such, it is necessary to average samples of an estimator for the ACF or power spectrum \citep{Diaz:2008co}.  For the ACF case, a covariance matrix between each lag estimate can be formed using the formulation in Equation 2 of \citet{hysell2008}, rewritten here as

\begin{equation}
\label{eqn:covcalc}
C_{\tau_1,\tau_2} = \frac{1}{2J} \left( \ R(0)  R^*(\tau_1-\tau_2) +  R(\tau_1) R^*(\tau_2) \right),
\end{equation}

\noindent where, $R(\tau)$ is the estimated ACF as a function of lag $\tau$, $C_{\tau_1,\tau_2}$ is the entry in the covariance matrix of the estimated ACF at lags $\tau_1$ and $\tau_2$,  and $J$ is the number of samples or pulses averaged together to create the estimate. The diagonals of this matrix can be thought of as the autocovariances of each of the lags.  Along these diagonals, by setting $\tau_2 = \tau_1 \equiv \tau$, Equation \ref{eqn:covcalc} simplifies to

\begin{equation}
\label{eqn:covdiag}
C_{\tau,\tau} = \frac{1}{2J} \left(  |R(0)|^2 +|R(\tau)|^2\right).
\end{equation}

The variance of the signal ACF estimate is further increased once sensor and sky noise is added.  If the noise is assumed to be uncorrelated with the signal, the error from the noise, $\left|R_w (\tau)\right|^2$ (e.g. the square of the noise ACF) can be added to the error from the inherent fluctuations in the signal, and the autocovariance expression becomes

\begin{equation}
\label{eqn:covdiagwn}
C_{\tau,\tau} = \frac{1}{2J} \left(  |R(0)|^2 +|R(\tau)|^2 + \left|R_w (0)\right|^2+\left|R_w (\tau)\right|^2\right).
\end{equation}

\subsection{Space-Time Errors}

For practical radar experiments, the use of waveforms with non-zero time extent introduces artifacts in the form of an associated radar ambiguity function \citep{nygren1996}.  The errors created through this ambiguity function lead to a blurring, or averaging, of ACFs from different points in time and space. This is analogous to a blurring operator one might encounter in standard image processing applications.
 
The space-time ambiguity, $L(\tau_s,\mathbf{r}_s,t_s,\tau,\mathbf{r},t)$, is the kernel of a Fredholm integral equation of the first kind operating on the ACF, $R(\tau,\mathbf{r},t)$, that can change over space, $\mathbf{r}$, and time $t$. The effect on the intrinsic ACF may be represented as follows,
%JOHN:  Is the above adjustment correct?   You had "It can b represented..."  which was vague
%Josh: Its better
 \begin{equation}
  \label{eqn:staf}
  \rho(\tau_s,\mathbf{r}_s,t_s) =\int L(\tau_s,\mathbf{r}_s,t_s,\tau,\mathbf{r},t)\:R(\tau,\mathbf{r},t)\ dV\ dt\ d\tau,
\end{equation}

\noindent where the subscript $s$ delineates the variable as the discretized version of its continuous counter part; i.e. the variable $\tau_s$ is the sampled version of the lag coordinate $\tau$. \citet{RDS:RDS20236} showed that the general ambiguity kernel is a separable function when the spatial coordinates are spherical, where ($r,\theta,\phi)$ represent, range, azimuth and elevation respectively.  In this case, Equation \ref{eqn:staf} becomes

\begin{equation}
\label{eqn:stafbrok}
\rho(\tau_s,\mathbf{r}_s,t_s)= \int G(t_s,t)F(\theta_s,\phi_s,\theta,\phi)\:W(\tau_s,r_s,\tau,r)\:R(\tau,\mathbf{r},t)\ dV\ dt\ d\tau,
\end{equation}

\noindent where $G(t_s,t)$ is the kernel for the time dimension, $F(\theta_s,\phi_s,\theta,\phi)$ is radar beam shape that acts as a kernel in azimuth and elevation, and $W(\tau_s,r_s,\tau,r) $ represents the range-lag ambiguity function, which acts as a kernel along range $r$ and lag $\tau$. 

For ISR experiment design, both the radar beam ambiguity and range-lag ambiguity functions create a fundamental trade off between statistical variation of the signal and spatiotemporal resolution of the signal. In order to  perform a statistical estimate of signal fluctuations, radar pulses need to be averaged together. This is necessary even for the case of infinite signal-to-noise ratio due to the inherent statistical nature of ISR  scatter.

The statistical averaging process is mainly done over time, but can be done over space as well by averaging across beams or range gates. If the ionosphere is drifting with a known vector velocity, \citet{RDS:RDS20236} pointed out that the stationarity assumption for ISR integration is  better met by averaging those beam positions that approximately correspond to the same scattering volume within the moving frame of reference; however, this approach may degrade spatial resolution at the expense of improvement in temporal resolution.  The optimal experiment design and processing strategy depend critically on the objectives of the experiment.   This trade space may be effectively explored through simulation.

%%%%%%%%%%%%%%%%%%%%%%%%%%%%%%%%%%%%%%%%%%%%%%%%%%%%%%%%%%%%%%%%%%%%%%%%%%%%%%%%%%%%%%%%%%%%%
\section{Simulation Methodology}

The SimISR software package allows one to analyze different  experiment scenarios by simulating the ISR measurement process. The space-time ambiguity is modeled through a three-dimensional blurring kernel along with appropriate coordinate transformations to account for target variation during radar acquisition.   The statistical error is taken into account by creating complex shaped Gaussian noise. In what follows we begin with a description of construction of spectral filters designed to create the noise-like signal received in ISR experiments. This is followed by description of the process of creating complex receiver voltage data. The last portion details the processing used to create statistical estimates of ACFs.

\subsection{Creating Filters}

The simulator takes as input a discretized set of ionosphere state parameters in Cartesian coordinates that can vary in time.  This corresponds to the true field we seek to reconstruct. The first step in the simulator is to create theoretical ISR spectra at each point from the prescribed parameters. For details on calculating these spectra from the intrinsic plasma parameters see, e.g., \citet{kudeki:milla:1} and \citet{kudeki:milla:2}. 

Once the spectra have been created, the simulator transforms the resulting values to a radar-centered spherical coordinate system. This coordinate change acts as a linear operator in spatial dimensions, and the spectra are accordingly weighted and averaged. The weighting in azimuth and elevation is determined by the antenna beam pattern, while the weighting in range (i.e. along beam) is simply a binary test of whether the spectra are within the range gate. If there are no spectra within the range gate, a nearest neighbor rule is used that selects the closest point in Cartesian space. 

The beam shapes needed for the antennas are created using standard theory for phased arrays and dish antennas. For the circular dish antennas the term $F(\theta_s,\phi_s,\theta,\phi)$ will be shown in a rotated frame system where the pointing positions $\theta_s$ and $\phi_s$ will be both zero 0 degrees in azimuth and elevation respectively and $\theta_r$ and $\phi_r$ will be $\theta$ and $\phi$ in this frame. The formulation for phased array antenna architectures is the following:

\begin{equation}
\label{eqn:circantenna}
F(\theta_r,\phi_r)=\frac{4r^2}{\lambda^2}\text{jinc}\left(\frac{2r}{\lambda}\sin{\phi_r}\right),
\end{equation}
where $r$ is the radius of the antenna in meters (m), $\lambda$ is the wavelength of the radar in meters (m) \citep{Blahut:2004wd}. The jinc function is defined as the following,
\begin{equation}
\label{eqn:jinc}
\text{jinc}(x)=\frac{J_1(\pi x)}{2x},
\end{equation}
where $J_1$ is the first order Bessel function of the first kind. The systems using phased array antennas use the basic formulation is the following:
\begin{equation}
\label{eqn:phasdarr}
F(\theta_s,\phi_s,\theta,\phi)= E_e^2(\theta,\phi)E_a^2(\theta_s,\phi_s,\theta,\phi)
\end{equation}
where $E_e$ is the antenna pattern of each element and $E_a$ is the array factor for pointing the beam \citep{Balanis:2005:ATA:1208379}. The full derivation of the theoretical beam pattern for the AMISR systems can be found in an appendix in \citet{RDS:RDS20236}.


The method to create the spectra for each point is an acceptable approximation because spatial correlations between the electron density fluctuations will be on the order of the Debye length \citep{farley1969} which is, in nearly all practical cases, significantly smaller than the beam width or range gate size. This is same as making the assumption of wide-sense stationarity with uncorrelated scattering (WSSUS)\citep{Kailath:1962jx}. The algorithm implementing spatial sampling is shown in the simplified diagram in Figure \ref{fig:beamdia}.

\begin{figure}[!t]
\centering
\includegraphics[width=2in]{beamsampling}
\caption{Each point is from a discrete sampling of a Cartesian space. The beams are broken up into range gates, separated by the dotted lines, and the parameters at each point overlapping within these gates are averaged.}
\label{fig:beamdia}
\end{figure}
 


Once the theoretical spectrum for a given scattering volume has been calculated, an appropriate spectral shaping filter is created. The method to create the filter given a desired spectrum or ACF can be done in a number of ways \citep{Kasdin:1995wi}. The implementation in SimISR creates an infinite impulse response filter in order to ensure a causal, minimum phase filter. 

The coefficients are determined using the ACF by solving the following set of equations,

\begin{equation}
\label{eq:filtereq}
\begin{bmatrix} R_m(0) & R_m(L-1)& \cdots & R_m(1) \\ 
R_m(1) & R_m(0)& \cdots & R_m(2)\\ 
\vdots & &\ddots  & \vdots \\  
R_m(L-1) & R_m(L-2) & \cdots & R_m(0) 
\end{bmatrix}
 \left[ \begin{array}{c} a_1\\ a_2\\\vdots \\ a_L \end{array} \right]=\left[ \begin{array}{c} R_m(1) \\ R_m(2)\\ \vdots \\R_m(L) \end{array} \right]
\end{equation}

\noindent where $R_m(l)$ are the ACF values, $L$ is the desired length of the filter, and $ a_i$ are the set of filter coefficients. The filter then takes the form in the frequency domain as the following,

\begin{equation}
\label{eq:filtz}
H_m(z) = \frac{G}{1-\displaystyle \sum_{l=1}^{L} a_l z^{-l}}.
\end{equation}
\noindent The gain term $G$ is used to make sure the noise has the correct variance. This can be calculated as 

\begin{equation}
\label{eq:gainterm}
G=\sqrt{\displaystyle \sum_{l=0}^L -a_l R_m(l)},
\end{equation}

\noindent where $a_0=-1$. This method has been used in similar ways in other contexts--e.g., the creation of vocoders for speech processing applications, as it creates causal and stable filters \citep{rabinerdigitalspeech}. Equivalently, this technique is creating an autoregressive (AR) process. Alternatively a moving average (MA) process could be use, which would result in a finite impulse response filter but calculating the coefficients for this filter can be much more computationally difficult \citep{kayvol1}.

\subsection{Simulated Complex Voltage Creation}

The algorithm used to create sampled complex receiver voltages employs a complex white Gaussian noise (CWGN) process (``plant") that is spectrally shaped at its output using a time domain filter. As stated in the previous subsection, each point in space and time will have a separate noise plant and filter which is derived from the plasma and radar parameters parameters. Figure \ref{fig:IQdiagram} presents a representative example. 

\begin{figure}[h!]
\centering
\includegraphics[width=4in]{diagrampart}
\caption{Diagram for main kernel of complex receiver voltage simulator signal flow.}
\label{fig:IQdiagram}
\end{figure}

The creation of one set of complex receiver voltage data can be represented by

\begin{equation}
\label{eq2}
y_m (i)= s(i)\left[h_m(i)*w(i)\right],
\end{equation}
 
\noindent where $s(i)$ is the overall transmitted pulse envelope, $h_m(k)$ is the time domain representation of the filter in Equation \ref{eq:filtz} and $w(i)\sim CN(0,\mathbf{I})$ or CWGN noise process. The pulse shape acts as a window function, since the plasma will only reflect energy during the time it is illuminated by the radar signal. 



After the data for each range gate $y_m(i)$ is created, the received signal's power spectrum can be calculated from ISR plasma scattering theory as 

\begin{equation}
\label{eq3}
P_r = \frac{\left(c\Delta T\right) G \lambda^2}{2(4\pi)^2}\frac{P_t }{R^2}\frac{\sigma_e N_e}{(1+k^2\lambda_D^2),(1+k^2\lambda_D^2 + T_r)},
\end{equation}
 
 \noindent where $P_r$ is the power received in Watts (W), $k$ is the wavenumber of the radar in radians per meters (rad/m), $c$ is the speed of light in m/s, $\Delta T$ is the along-range gate extent in seconds, $G$ is the gain of the antenna, $P_t$ is the power of the transmitter in W, $\sigma_e$ is the electron radar cross section in $m^2$,  $\lambda_D$ is the Debye length in m, $N_e$ is the electron density in m$^{-3}$, and $T_r$ is the electron to ion temperature ratio.
  
The received signal power calculated at each range gate using Equation~\ref{eq3} is used as a scaling constant for each $y_m(i)$ series.  A delayed and summed operator yields a model of the received radar scatter signal:
 
\begin{equation}
\label{eq4}
x(n) = \displaystyle\sum\limits_{m =0}^{M-1} \alpha(m)y_m(n-m),
\end{equation}

\noindent where $\alpha(m) = \sqrt{P_r(m)}/\widehat{\sigma}_y$ and $\sigma_y$ is the standard deviation of $y_m(i)$ and the bracket denotes an estimator. Each signal from each $M$ number of range gates is assumed independent of one and other as this would violate the assumption that any spatial correlations drop off much faster than the distance covered by one range gate, see \citet{RDS:RDS20236}. Lastly, to model total noise from the radar system and environment, an additive CWGN process is included, creating the final simulated complex receiver voltage sequence

\begin{equation}
\label{eq:addnoise}
x_f(n) = x(n) +\sqrt{\frac{k_bT_{sys}B}{2}} w(n), \quad w(n)\sim CN(0,\mathbf{I})
\end{equation}

\noindent where $k_b$ is Boltzmann's constant, $T_{sys}$ is the system temperature and $B$ is the system bandwidth.
A full diagram of the model can be seen in Figure \ref{fig:isrdiag}.

\begin{figure}[!h]
\centering
\includegraphics[width=7.0in]{diagram}
\caption{Full SimISR signal flow diagram where the diagram from Figure~\ref{fig:IQdiagram} is replicated for each range gate. The signals for each range gate are then weighted and then summed together to form the data from a single beam. Also note that each noise plant is assumed independent of one and other.}
\label{fig:isrdiag}
\end{figure}

\subsection{ACF Estimation}
\label{sec:acf}
After complex receiver voltage data have been created, the data are processed to create estimates of the ACF at desired points of space \citep[see, e.g.,][]{farley1969,nygren1996}. This processing follows the flow chart presented in Figure \ref{fig:chain}.  Note that we assume here a signal pipeline which creates a single altitude measurement for analysis.  More sophisticated approaches for ISR analysis exist that use information from multiple altitudes, including full profile analysis \citep{RDS:RDS3308}, lag profile inversion \citep{Virtanen:20082vx}, and others, but treatment of these approaches is beyond the scope of this work.

\begin{figure}[!t]
\centering
\includegraphics[width=6in]{datastackchain}
\caption{ISR signal processing chain, with signal processing operations as squares and data products as diamonds.}
\label{fig:chain}
\end{figure}


The lag product formation is an initial estimate of the autocorrelation function. The sampled complex receiver voltage can be represented as $x(n) \in\mathbb{C}^N$ where $N$ is the number of samples in an inter-pulse period. For each range gate $m\in 0,1,...M-1$ a complex autocorrelation is estimated for each lag of $l \in 0,1...,L-1$.  To get better statistics this operation is performed for each pulse $j\in 0,1,...J-1$ and then summed over $J$ independent pulses. The entire operation to form the initial estimate of $\widehat{R}(m,l)$ may be expressed as

\begin{equation}
\label{eq:lagpro}
\widehat{R}(m,l) = \displaystyle\sum\limits_{j=0}^{J-1} x(m-\lfloor l/2\rfloor,j)x^*(m+\lceil l/2 \rceil,j).
\end{equation}

The case shown in Equation \ref{eq:lagpro} is a centered lag product.  Other types of lag product calculations are available but a centered product is most common. For this case, the range gate index $m$ and sample index $n$ can be related by $m=n-\lfloor L/2\rfloor$ and the maximum lag and sample relation is $M=N-\lceil L/2 \rceil$.  This lag product formation is the first step in computing a discrete Wigner Distribution \citep{TFAcohen}. This  step adds a bias to the ACF estimate which acts as a weighting on larger lags, represented as $\mathcal{W}(l)$ where weighting can be calculated from details of the range-lag ambiguity function. The expected value for the estimator, assuming the use of a simple uncoded pulse waveform, becomes

\begin{equation}
\label{eq:lagprobias}
\left\langle\widehat{R}(m,l) \right\rangle = \mathcal{W}(l)R(m,l) =\frac{L-l}{L}R(m,l).
\end{equation}


Applying a summation rule is generally the next step in creating an estimate of the autocorrelation function for single altitude analysis. This is done for a number of reasons, but primarily to improve estimate statistics.  Furthermore, if the right rule is chosen, then the range ambiguity can be made approximately constant across the lags, which can make inversions easier \citep{nygren1996}. Summation rules based on other criteria can be used but our simulations use the trapezoidal summation, which is a common choice and leads to uniform range resolution across all lags. It can be represented as follows:

\begin{equation}
\label{eq:sumrule}
\widehat{R}_s(m,l) = \displaystyle\sum\limits_{i=-((v-1)/2+\lceil l/2 \rceil)}^{((v-1)/2+\lfloor l/2\rfloor)} \widehat{R}(m+i,l),
\end{equation}

\noindent where $v$ is the 'volume' index or the number of gates integrated at zero lag (restricted to positive odd integers here) and $\widehat{R}_s(m,l)$ is the final ACF estimate after the summation rule \citep{nygren1996}. 
% pcom checked for variables, already defined
However, the final result of this summation rule will still lead to a statistically biased ACF. For the uncoded waveform case, this summation rule leads to the following expected value for the estimator \citep{nygren1996},

\begin{equation}
\label{eq:sumruleest}
\left\langle\widehat{R}_s(m,l) \right\rangle  =\frac{v+l}{v\mathcal{W}(0)}\mathcal{W}(l)R(m,l) =\left(-\frac{1}{vL}l^2+\frac{L-v}{Lv}l+1\right)   R(m,l).
\end{equation}

\noindent Also of note, the function in the parenthesis of the RHS of Equation~\ref{eq:sumruleest} is also known in the signal processing literature as a window function along the lags of the ACF \citep{dtsp:openhiem}. This window function, along with its frequency content, can be seen in Figure~\ref{fig:isrwindow}.

\begin{figure}[!t]
\centering
\includegraphics[width=5in]{ISRWindow}
\caption{The window function that is applied to the ACF as seen in Equation~\ref{eq:sumruleest} in the left pane for $L=15$ and its 128 point length Fast Fourier Transform (FFT) in the right panel. }
\label{fig:isrwindow}
\end{figure}


Finally, noise effects are included by subtracting an estimate of the noise correlation from $\widehat{R}_s(m,l)$.  We represent the noise correlation function as $\widehat{R}_w(m,l)$, the ACF estimate of the background noise process of the radar $w(n_w)$ using the steps in Equation \ref{eq:lagpro} and Equation \ref{eq:sumrule}. In a real radar system the noise process is typically sampled either during a calibration period for the radar when nothing is being emitted, or at ranges sufficiently distant that scattered ionospheric signal is assumed to be negligible. If there are localized but large returns, for example coherent reflections from satellites, this impact can be reduced by applying estimators that use order statistics \citep{ordstatcfar}. For our simulated estimate, we simply use a noise process with the same correlation structure and power as the noise that was added. The final estimate of the autocorrelation function after the noise subtraction and summation rule is represented by $\widehat{R}_f(m,l)$. 


After the final estimation of the spectrum is complete, nonlinear least squares fitting takes place to determine plasma parameters. The class of nonlinear least squares problems relevant to ISR parameter estimation can be represented as the minimization of a cost function of the form \citep{kayvol1},

\begin{equation}
	\mathbf{\hat{p}}= \underset{\mathbf{p}}{\text{argmin}} (\mathbf{y}-\bm{\theta}(\mathbf{p}))^*\mathbf{C}_{\mathbf{y}}^{-1}(\mathbf{y}-\bm{\theta}(\mathbf{p})).
\label{nlls}
\end{equation}

In Equation \ref{nlls}, the data represented as $\mathbf{y}$ would be the final estimate of the ACF $\widehat{R}_f(m,l)$ at a specific range, or its spectrum $\widehat{S}_f(m,\omega)$. The matrix $\mathbf{C}_{\mathbf{y}}$  is the covariance matrix from the ACFs or spectra depending on what is being fit. The covariance matrix for the ACF is detailed in Equation \ref{eqn:covcalc}, while the covariance matrix of the spectra is simply the ACF matrix but with discrete Fourier Transforms applied to the rows and columns. The parameter vector $\mathbf{p}$ would be the plasma parameters $N_e$, $T_e$, $T_i$ and $V_i$. The fit function, $\bm{\theta}$, is the IS spectrum calculated from a model \citep[e.g.,][]{kudeki:milla:1} and smeared by the ambiguity function.  In the case of the long pulse and a system with a large receiver bandwidth compared to the spectra, the ambiguity can be simply applied by multiplying it with the autocorrelation function $R(l)$ assuming the proper summation rule used. As in previous publications, \citep[e.g.,][]{nikoukar2008}, the Levenberg-Marquardt algorithm is used to fit the plasma parameters to the ACFs or spectra \citep{levenberg1944,marquardt:1963}.

The last step is to calculate the errors in the parameter estimates. In order to do this a numerical approximation is computed of the Jacobian matrix between the data and the ACF, $\mathbf{J}$, at $\mathbf{p}=\mathbf{\hat{p}}$. Given this Jacobian, the formula to estimate the parameter error matrix, $\mathbf{C}_{\mathbf{\hat{p}}}$ according to \citet{Hysell:2000cq}, is


\begin{equation}
\label{eqn:jacinv}
\mathbf{C}_{\mathbf{\hat{p}}}=(\mathbf{J}^T \mathbf{C}_{\mathbf{y}}^{-1}\mathbf{J})^{-1}.
\end{equation}


\noindent  The variances of the parameters are then taken as the diagonals of the matrix.  (For the purposes of this study, we neglect the off-diagonal variances representing correlation between errors in plasma parameters, but this information is available within the SimISR output.)




%%%%%%%%%%%%%%%%%%%%%%%%%%%%%%%%%%%%%%%%%%%%%%%%%%%%%%%%%%%%%%%%%%%%%%%%%%%%%%%%%%%%%%%%%%
\section{Simulation Examples}
The framework for SimISR allows exploration of a number of aspects of ISR processing. Within the scope of this article, we will focus on four application examples.

The first example demonstrates how the simulator can be used for Monte Carlo estimates of ISR spectra. In this case, we hold all of the plasma parameters constant and determine how the distribution of the measured parameters evolve. The next example uses a simple altitude distribution of ionospheric plasma parameters to show the impact of the forward model of the ISR on a basic measurement of electron density. This is intended to illustrate that basic ambiguities inherent in ISR measurements can give the appearance of a change in morphology of the plasma phenomena when none truly exists. Finally, the output of a fully consistent multi-fluid ionosphere model is used as input to the SimISR and is applied in two use cases relevant to experiment planning, one varying over two spatial dimensions and another varying in all three spatial dimensions. The results of these use cases illustrate an inherent tradeoff in experiment construction between reducing statistical fluctuations in the measurement and increasing distortion in the final reconstruction. 

\subsection{Monte Carlo Example}

It is often necessary to obtain a large number of sensor measurements for a statistical study or for creating of a training data set for a pattern recognition algorithm. This can be a very burdensome search and classification task for the researcher if the input set must be drawn from actual sensor measurements. However, a number of useful cases exist where SimISR can be profitably employed to create a synthetic data set instead, saving considerable work over case assembly from a real ISR data based training set.  We explore one such example in this section.


For this example we show how distributions of plasma parameter measurements change as more pulses are averaged. To do this we created a field of constant plasma parameters typical of the high latitude ionosphere at around 250 km, and performed a Monte Carlo-type simulated statistical experiment using a number of independent realizations. We use the parameters for the Poker Flat AMISR system for this simulation \citep{Valentic:2013jg}, along with the plasma parameter listed in Table \ref{tb:param1}. For a number of independent radar pulse counts, $J$, we used 4,600 realizations of the statistical ISR measurement process in each case to create statistical distributions of measured parameter values. The histograms are created using each of these relations, although some come from the same beam and thus there can be some statistical correlation. The distributions can be seen in Figure \ref{fig:statshistall} which show distributions where 200, 500 and 1000 pulses are used respectively. For a given pulse count $J$, the plasma parameters have a Gaussian-like distribution. As expected the distribution narrows as the number of pulses $J$ is increased. Another observation is that as the number of pulses increases the bias and skew in the measurement is reduced. This is likely due to closer convergence to a Gaussian like distribution as more pulses are added. The electron density measurement is proportional to the 0$^{th}$ lag of the ACF, along with a multiplication by $1+T_r$. This would cause the distribution to become more Gaussian like as more pulses are added through the Central Limit Theorem \citep{papoulis2002}. Also of note, these histograms are 1-D projections from a 4-D distribution that likely has a very elaborate correlation structure, due to the relationship between the ACF and plasma parameters being very mathematically complex. 


\begin{table}[!t]
\centering
\caption{Simulation parameters.}
\label{tb:param1}
\begin{tabular}{ll}
Species & O+ e-\\
$N_e$    & $1\times 10^{11}$ \\
$T_e$      & $2100^o$ K   \\
$T_i$      & $1100^o$ K \\
$V_i$      & $0$ m/s
\end{tabular}
\end{table}

\begin{figure}[!t]
\centering
\includegraphics[width=5in]{datahist}
\caption{Normalized histograms of fitted plasma measurements from cases with 200, 500 and 1000 pulses integrated. These are estimates of probability density functions for each of the plasma parameters measurements given the values in Table \ref{tb:param1}.}
\label{fig:statshistall}
\end{figure}

ISR error analysis also benefits from SimISR's ability to generate a large number of samples of fitted parameters. In particiular, ISR measurements need to have estimates of the errors, and the accuracy of the estimates of these errors can be explored using SimISR. Using the 1000 pulse case seen in Figure \ref{fig:statshistall}, we can compare simulated output distributions with the actual distribution of parameter values. Figure \ref{fig:statshistsingle} shows a comparison of these two different models using. The first method, represented by the blue line, uses the sample mean and variance calculated from each as the variance and mean for a Normal distribution. The other method, that generated the green line, calculates an average squared error from the true value for each parameter for the variance in a Normal distribution and uses the true value for the mean. 
%% PJE: I'm confused by the logic of this sentence above (also "shows gives" etc?) - please reword.
%% JPS: Just changed 
This example shows that the parameter distributions are well represented by a Gaussian function but that the error estimated from the fit may not give a completely accurate representation of variance of these parameters. 

\begin{figure}[!t]
\centering
\includegraphics[width=5in]{histsingle}
\caption{Distributions of fitted plasma measurements from cases with 1000 pulses integrated. The red curve shows the actual distribution derived from a histogram of 4600 measurements. The blue curve is a Normal distribution using the MSE from the measurements as the variance and the average parameter value as the mean.  The green curve is a Normal distribution using the average estimate of error squared that comes with the parameter measurement as the variance and the average parameter value as the mean.}
\label{fig:statshistsingle}
\end{figure}


The SimISR tool is useful for identifying situations where assumptions in the parameter fitting break down. For example, a number of studies have explored the case where ISR parameter measurements and spectra show evidence of non-Maxwellian plasma behavior \citep[see, e.g.,][for AMISR examples]{Akbari:2012dz,Akbari:2015fv}. Future studies using the simulator could help create a training set that can be used with a pattern recognition algorithm to identify cases where normal fitting procedures may be incorrect due to violation of Maxwellian assumptions.

\subsection{Electron Density Measurement}
An important aspect of experiment design is determining the observability of plasma phenomena with ISR. The simulator can be used to help understand the trade space accompanying a given experimental configuration. With this in mind, we use a simple two dimensional spatial field of ionospheric parameters as an illustrative case study. An all-O$^+$ ionosphere is created with a background electron density that follows a Chapman function with $1\times10^{11}$ m$^{-3}$ as the peak value and a constant electron and ion temperatures of 2000 K and 1500 K respectively. The background ionosphere is depicted in Figure \ref{fig:background1}.

\begin{figure}[!t]
\centering
\includegraphics[width=4in]{backgroundandsamp}
\caption{Contour of background $N_e$ for simulations and the spatial sampling pattern overlaid as white dots.}

\label{fig:background1}
\end{figure}


We first explore how a thin stationary density enhancement is resolved with the radar beam pattern shown in Figure \ref{fig:background1}, where each dot is a range gate in one of the 25 beams used. In Figure~\ref{fig:stationaryall}a, a thin density enhancement 2 km in width and 5 times the background is placed in the radar field of view. The enhancement is at the resolution limit of the original Cartesian grid, a delta function in the x direction.  The fitted electron density results from our simulation, without any additional averaging beyond the processing described in Section \ref{sec:acf},

are seen in Figure \ref{fig:stationaryall}b and c using 15 and 60 second integration times, corresponding to 60 and 240 pulses per position, respectively. The different integration times show that, although the enhancement is blurred, the variance of the measurement impacts the quality of the reconstruction because of the inherent noise-like nature of the signal. The expected errors for both of the reconstructions are shown in Figure \ref{fig:errorstationaryall}. As expected, the estimated errors show that uncertainties are larger for the case with fewer pulses (i.e. smaller ensemble average size). 

Because we know the input parameters, we can do a quick comparison using the root mean squared error (RMSE) for each case. By comparing the RMSE between the 15 and 60 second integration cases, we find that the ratio between the two cases is approximately 5.4 in the simulation output.  However, the expected RMSE ratio between the two should be approximately 2 since the variance of the ACFs scales as $1/\sqrt{J}$, where $J$ is the number of pulses. If we use a median instead of a mean operator in the error calculation, this ratio becomes 1.55, more in line with the expected statistical error scaling, largely due to the large outliers being disregarded.  Further investigation of the quantitative error discrepancy is a larger effort and beyond the scope of this study; however, in general, the ratios between the errors and expected errors are relatively close when employing a median estimator that inherently rejects large outliers.

\begin{figure}[!t]
\centering
\includegraphics[width=6in]{stationary}
\caption{Results of stationary enhancement simulation. a) Input $N_e$. b) Output of simulator with 15 second integration. c) Output of simulator with 60 second integration.}
\label{fig:stationaryall}
\end{figure}

\begin{figure}[!t]
\centering
\includegraphics[width=4in]{Errorstationary}
\caption{Standard deviations represented as percentages compared to the values in Figure \ref{fig:stationaryall}. a)  Error percentages from fit for 15 second integration example. b) Error percentages from fit for 60 second integration example.}
\label{fig:errorstationaryall}
\end{figure}

The blurring effect seen in this case study is not constant throughout the simulated space due to the way the radar samples the space. This is illustrated in Figure \ref{fig:moving10mins} and Figure \ref{fig:moving14mins}, where as the enhancement moves through the scene at 500 m/s, its apparent size is affected by the orientation of the radar beams. 

As the enhancement becomes parallel to the radar beams, close to 0 km in the X plane,

its morphology in the reconstruction becomes smaller along the X-axis, as the range ambiguity is much larger than the cross range ambiguity. (To see the input and output electron density change with time, the reader is referred to Movies S1 and S2.). In both cases the expected errors, shown in panel c in both Figure \ref{fig:moving10mins} and Figure \ref{fig:moving14mins}, give us confidence in these results as they are much lower value than the enhancements and background.

\begin{figure}[!t]
\centering
\includegraphics[width=6in]{moving6mins}
\caption{Results of moving enhancement simulation at 600 seconds. a) Input $N_e$. b) Output of simulator with 60 second integration. c) Estimated error percentages from fit.}
\label{fig:moving10mins}
\end{figure}


\begin{figure}[!t]
\centering
\includegraphics[width=6in]{moving14mins}
\caption{Results of moving enhancement simulation at 840 seconds. a) Input $N_e$. b) Output of simulator with 60 second integration. c) Estimated error percentages from fit.}
\label{fig:moving14mins}
\end{figure}

This change in the shape of the enhancement can give the impression that its morphology has evolved as it was moving through the field of view of the radar. This could lead to an incorrect interpretation of the physical process taking place, and issue raised by \citet{Dahlgren:2012dq}.   Thus, one must be careful when analyzing these sorts of reconstructions.

Lastly, for this type of simulation, we show an example to demonstrate the situation where a set of two different input parameters can yield qualitatively similar results. For these cases we create electron density enhancements similar to those seen in the high latitude observational study of \citet{Semeter:2005fo} during a poleward boundary intensification event. The sizes of these enhancements are 10 km width and 18 km width. The enhancement in the 10 km width example is 6 times higher than the background while the 18 km width enhancement is 3 times higher than the background.

The input electron density, the fitted electron density and the expected error for the 10 km enhancement can be seen in Figure \ref{fig:moving10all}. The same images for the 18 km wide case can be seen in Figure \ref{fig:moving18all}. Both cases show that electron density enhancements are well above the expected errors.

The fitted electron density for both the 10 km enhancement and 18 km enhancement cases show nearly identical results. This simple example demonstrates the possibility to create a non-unique solution in ISR experiments. The results further demonstrate that SimISR can provide useful information in this case, in that it can provide information during the design phase of an experiment that highlight possibilities for ambiguous observational results between two different sets of phenomena. (For the full input electron density changing with time for the 10 km example, see Movie S3 and for the fitted data S4. The 18 km case can be seen in Movies S5 and S6 for input and fitted parameters respectively.)

\begin{figure}[!t]
\centering
\includegraphics[width=6in]{moving10kminouterr}
\caption{10 km wide enhancement moving simulation at 480 seconds. a) Input $N_e$. a)  b) Fitted $N_e$ with 60 second integration. c) Estimated error percentages from fit.}
\label{fig:moving10all}
\end{figure}

\begin{figure}[!t]
\centering
\includegraphics[width=6in]{moving18kminouterr}
\caption{18 km wide enhancement moving simulation at 480 seconds. a) Input $N_e$.  b) Fitted $N_e$ with 60 second integration. c) Estimated error percentages from fit.}
\label{fig:moving18all}
\end{figure}

\subsection{Full Parameter Experiment}
\label{sec:fullparam}
Another SimISR use case employs input plasma parameters derived from the multi-fluid model developed by \citet{semeter:plasmatransport2012}. The specific model run was originally used by \citet{Perry:2015jf} to assist in interpreting measurements of polar cap arcs from the Resolute Bay Incoherent Scatter Radar (RISR). Images of the modeled plasma parameters are shown in Figure \ref{fig:plparamst0} and Figure \ref{fig:plparamst60}. The enhancements in electron density, electron temperature, and ion temperature comprise the self-consistent response of the ionosphere to a field-aligned current system with amplitude .875 $\mu$A/m$^2$.  The source is made to move with respect to the radar at a velocity of 200 m/s (a  value inferred from optical forms observed during this event).  A channel of soft electron precipitation (50-500 eV in energy) is added to the upward current channel, with energy flux consistent with the amount of electron heating seen during the event.  A reasonable objective for a multi-beam ISR experiment could be to validate model predictions of conditions leading to the arc-adjacent density depletion seen in panel a of Figure \ref{fig:plparamst60}.  In this specific case SimISR can be used to assess observability of dynamic plasma structuring and establish confidence intervals on the ISR results. 


\begin{figure}[!t]
\centering
\includegraphics[width=6in]{backgroundallparams}
\caption{Background ionospheric parameters ($N_e$, $T_e$, $T_i$) along with number density of ion species, used for simulations.}
\label{fig:plparamst0}
\end{figure}

\begin{figure}[!t]
\centering
\includegraphics[width=6in]{0960_15_int}
\caption{Perturbations to Figure \ref{fig:plparamst0} due to an imposed current system of .875 $\mu$A/m$^2$ at $t=960$ s, representing a polar cap auroral arc that is sweeping across the field-of-view of the radar at about 200 m/s \citep[c.f.][]{Perry:2015jf}.}
\label{fig:plparamst60}
\end{figure}

Using a beam pattern similar to the one seen in the right panel of Figure \ref{fig:background1}, we use SimISR to explore how an electronically scanned ISR may reconstruct this dynamic auroral arc system.  For the purposes of illustration, we use the contrived case where the radar spatial beam pattern is defined to be in the plane of convection.  It is also assumed that the ion species present (particularly their individual masses and relative concentrations) are known. This is a common use case in ISR fitting, as allowing ion concentrations to be free parameters can, in some situations, allow for non-unique solutions depending on the ion species that are present.  If the fixed, a priori composition ratios between the different ion species are incorrect, this can lead to errors in the final parameter estimates, with ion temperature particularly affected. For the lower F region ionosphere, parameter distortions can occur between 150-250 km where the ionosphere changes from NO$^+$ dominated to O$^+$ dominated \citep{Zettergren:2011ej, Blelly:2010gf}. We also note that as the field aligned current passes through the simulated field, an influx of NO$^+$ appears in the region of rapid ion mass transition for this simulation, potentially violating ion composition assumptions \citep{Perry:2015jf}.


The output of SimISR in the auroral arc case can be seen in Figure \ref{fig:fplparamst60} and in Movie S8. The integration is started at $t=960$ s into the multi-fluid simulation with the plasma parameters shown in Figure \ref{fig:plparamst60}. The full set of plasma parameters over time can be seen in Movie S7. For this case, a 60 second integration time is used, which for the 27 beam radar experiment set up gives 255 pulses per position. These plasma parameters are linearly interpolated to a Cartesian grid and plotted using the GeoData API \citep{john_swoboda_2016_154533}. Lastly, the expected errors from the fit can be seen in Figure \ref{fig:fplparamst60err}, which are of much lower value that the fitted parameters.

We highlight several features in the fitted results. First, the predicted enhancements in electron and ion temperature are clearly observable and well above the expected error. Second, we examine whether the predicted density cavity in the downward field-aligned current region is  detectable. A deepening and broadening region of plasma evacuation is predicted as a self-consistent response to a confined up-down current pair \citep[e.g.,][]{cran;cavity}.  But it has been unclear whether this prediction can be validated with ISR, since it involves detecting organized channels of reduced backscatter power embedded within a higher density background.

The images shown in Figure \ref{fig:fplparamst60} represent the best case scenario for identifying the presence of this cavity since, at this time, the cavity is nearly co-aligned with one of the beams. Using density measurements alone (panel a) the presence of the cavity is visible, but only marginally so, as it is blended with the adjacent enhancement produced by the applied precipitation in the upward current channel.  A similar ambiguity exists with the electron temperature result, which could easily be interpreted as purely an effect of heating from soft precipitation. However, the ion temperature increase in panel b is decidedly narrower than the electron temperature enhancement, offering a possible observable fingerprint for the presence of a confined up-down current pair.  The simulation result illustrates the efficacy of a collective analysis of all plasma state parameters in evaluating the physical mechanism responsible for an observed dynamic. This is a common approach in data assimilation problems.

%% Fitted Data

\begin{figure}[!t]
\centering
\includegraphics[width=6in]{0960_60_int}
\caption{Fitted Plasma Parameters for the auroral arc case at $t=960$ s with 60 second integration.}
\label{fig:fplparamst60}
\end{figure}

\begin{figure}[!t]
\centering
\includegraphics[width=6in]{0960_60_int_err}
\caption{Estimated error percentages from fitted plasma parameters for the auroral arc case at $t=960$ s with 60 second integration.}
\label{fig:fplparamst60err}
\end{figure}

\subsection{Full 3-D Reconstruction}

A common application of ESA-based ISR's (PFISR and RISR) is to create three-dimensional time-dependent visualizations of  dynamically evolving parameter fields \citep[e.g.,][]{Nicolls:2007ie,Semeter2009738,semeter:jgr2010,dahlgren2012di}.   Such results are visually compelling but, as yet, there has been no framework advanced to evaluate uncertainties and potential artifacts in these interpolated views, or to apply these results in a quantitative comparisons with predictions from physical models.  The state of modeling of the coupled magnetosphere-ionosphere system has progressed considerably in recent years. In particular, hybrid fluid-kinetic models are able to make detailed  predictions of small- and meso-scale processes underlying global-scale system behavior \citep[e.g.,][]{damiano;jgr2005,semeter:plasmatransport2012,akbari:jgr2016}.  Simulation will enable us to do formal hypothesis testing on these predictions, which often involves the detection of subtle space-time variations in a parameter field.

As an example, SimISR was driven using plasma parameters computed from a three-dimensional version of multi-fluid model \citep{zettergren2015dynamics} used to test simISR in the previous section. The parameter distributions represent the self-consistent ionospheric response to a 0.65-$\mu$A/m$^2$ field-aligned current (FAC) that is turned on at model time $t=0$, and remains constant in time.  The Region 1-like FAC creates large enhancements in ion temperature due to frictional heating, and also causes smaller amplitude enhancements (and depletions, as before) in electron density.

The spatial domain is sampled using the beam pattern and sampling lattice shown in Figure \ref{fig:3dsampling}. This is a 121 beam pattern similar to the one used by \cite{Semeter:2008hs}. For this simulation the data are integrated over 315 seconds which, with a 8.7 ms IPP and the current beam position, yields 300 pulses per position.

After the data were fitted, a three-dimensional natural neighbor interpolation was performed \citep{Sukumar:nn2001} and the results

plotted using the GeoData API \citep{john_swoboda_2016_154536}.

\begin{figure}[!t]
\centering
\includegraphics[width=6in]{Sampling3d}
\caption{The beam pattern used in angle space and the resulting 3-D spatial sampling pattern used for the three dimensional SimISR use case simulation.}
\label{fig:3dsampling}
\end{figure}

The input plasma parameters and the results of SimISR simulation are summarized in Figure \ref{fig:3dparams}. The results of the full simulation can be seen in Movie S9. For the selected configuration, the reconstructed density and ion temperature fields capture the predicted spatial variations reasonably well, providing confidence that this experimental configuration would yield a positive detection of the spatial variations predicted by the applied FAC.  Reconstruction of the electron temperature enhancement (which is derived from a higher-order moment of the ISR spectrum) is less conclusive. This highlights the important point that variance is parameter dependent. In this case, SimISR informs us that, for this FAC and this experimental configuration, a negative result for $T_e$ comparison is not sufficient grounds to discount the model prediction.  A further refinement of the experiment might yield a positive result.
\begin{figure}[!t]
\centering
\includegraphics[width=6in]{60_60}
\caption{Input and output of full 3-D reconstruction of plasma parameters.}
\label{fig:3dparams}
\end{figure}

The sampling pattern picked for this simulation was chosen to get as dense a set of non-overlapping beams as possible.  This has been the general objective for several previous volumetric imaging experiments. The overall phenomena varies over a much larger area, so there is an obvious trade off between sampling density and the support region. These sorts of simulations can help experiment planners to understand these trade offs and possibly yield to innovative sampling stratagems for specific phenomena. 
%%%%%%%%%%%%%%%%%%%%%%%%%%%%%%%%%%%%%%%%%%%%%%%%%%%%%%%%%%%%%%%%%%%%%%%%%%%%%%%%%%%%%%%%%%

\section{Conclusion}
We have constructed SimISR, a simulation tool encapsulating the full ISR measurement process, incorporating ESA radar capabilities and the full radar space-time ambiguity along with inherent ISR error sources. Possible uses for SimISR in the research community have also been discussed and examples have been shown. These examples show how one can use the simulator to create large statistical data sets and also to more optimally design ionospheric radar experiments in the ISR space within the inherently large number of free parameters afforded by the radar control parameters. 

In the future, SimISR development will continue by adding new radar waveform modes, as currently only Barker code and uncoded single pulse modulations are available at this time. The simulator can also be used to create synthetic data for traditional single antenna-based system design applications. Other possible expansions of the simulator include capability to calculate returns from each receiver element in a ESA based ISR system, such as the planned architecture of EISCAT-3D, along with multi-static radar capabilities. These future additions will increase the simulator's value to designers who wish to more optimally exploit the capabilities of new systems. 

A more immediate application of the SimISR tool can aid researchers in experiment planning. There are a number of phenomena that change on very small spatio-temporal scales, e.g. at high latitudes, and capturing observations of them would greatly benefit from optimization of the experiment set up. Researchers can use SimISR to iterate through different set ups for their experiments, as opposed to a heuristic selection of a single observational approach where prior use is the sole design factor. The ability of SimISR to directly create complex receiver voltage data provides a significant and novel capability, as some geophysical phenomena, such as those that occur at time scales on the order of an interpulse period, can only be fully explored at this data level. This could lead to researchers coming up with new ways to analyze data, such as novel integration schemes to resolve phenomena at small spatio-temporal scales.

By releasing this simulation to the community, we hope that other researchers can find utility in it as they plan experiments or analyze observed radar data post-experiment.

SimISR is written as a Python module that can be accessed at https://github.com/jswoboda/SimISR. This website contains code and instructions on how to install the software along with reference use case examples.
%%%%%%%%%%%%%%%%%%%%%%%%%%%%%%%%%%%%%%%%%%%%%%%%%%%%%%%%%%%%%%%%%%%%%%%%%%%%%%%%%%%%%%%%%%
\begin{acknowledgments}
This study was supported by the U.S. National Science Foundation (NSF), through Aeronomy Program Grant AGS-1339500 to Boston University, Cooperative Agreement AGS-1242204 between NSF and the Massachusetts Institute of Technology, and by the Air Force Office of Scientific Research under contract FA9550-12-1-018.   Work at ERAU was supported by NSF grant AGS-1339537.  The authors are grateful to the International Space Science Institute (ISSI, Bern, Switzerland) for sponsoring a series of workshops from which the idea for this work emerged. 

Software used to create figures for this publication can be found at https://github.com/jswoboda/. Please contact the corresponding author, John Swoboda at swoboj@bu.edu, with any questions regarding the software along with any requests for specific data used for the figures. \end{acknowledgments}


\bibliographystyle{BibTeX/agufull08}
\bibliography{BibTeX/litreview}
\end{article}

\end{document}

